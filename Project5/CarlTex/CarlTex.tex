\documentclass[reprint,english,notitlepage]{revtex4-2}  % defines the basic parameters of the document

% if you want a single-column, remove reprint

% allows special characters (including æøå)
\usepackage[utf8]{inputenc}
\usepackage[english]{babel}
\usepackage{float}
%% note that you may need to download some of these packages manually, it depends on your setup.
%% I recommend downloading TeXMaker, because it includes a large library of the most common packages.

\usepackage{physics,amssymb}  % mathematical symbols (physics imports amsmath)
\usepackage{graphicx}         % include graphics such as plots
\usepackage{xcolor}           % set colors
\usepackage{hyperref}         % automagic cross-referencing (this is GODLIKE)
\usepackage{tikz}             % draw figures manually
\usepackage{listings}         % display code
\usepackage{subfigure}        % imports a lot of cool and useful figure commands

% defines the color of hyperref objects
% Blending two colors:  blue!80!black  =  80% blue and 20% black
\hypersetup{ % this is just my personal choice, feel free to change things
    colorlinks,
    linkcolor={red!50!black},
    citecolor={blue!50!black},
    urlcolor={blue!80!black}}

%% Defines the style of the programming listing
%% This is actually my personal template, go ahead and change stuff if you want
\lstset{ %
	inputpath=,
	backgroundcolor=\color{white!88!black},
	basicstyle={\ttfamily\scriptsize},
	commentstyle=\color{magenta},
	language=Python,
	morekeywords={True,False},
	tabsize=4,
	stringstyle=\color{green!55!black},
	frame=single,
	keywordstyle=\color{blue},
	showstringspaces=false,
	columns=fullflexible,
	keepspaces=true}


%% USEFUL LINKS:
%%
%%   UiO LaTeX guides:        https://www.mn.uio.no/ifi/tjenester/it/hjelp/latex/ 
%%   mathematics:             https://en.wikibooks.org/wiki/LaTeX/Mathematics

%%   PHYSICS !                https://mirror.hmc.edu/ctan/macros/latex/contrib/physics/physics.pdf

%%   the basics of Tikz:       https://en.wikibooks.org/wiki/LaTeX/PGF/TikZ
%%   all the colors!:          https://en.wikibooks.org/wiki/LaTeX/Colors
%%   how to draw tables:       https://en.wikibooks.org/wiki/LaTeX/Tables
%%   code listing styles:      https://en.wikibooks.org/wiki/LaTeX/Source_Code_Listings
%%   \includegraphics          https://en.wikibooks.org/wiki/LaTeX/Importing_Graphics
%%   learn more about figures  https://en.wikibooks.org/wiki/LaTeX/Floats,_Figures_and_Captions
%%   automagic bibliography:   https://en.wikibooks.org/wiki/LaTeX/Bibliography_Management  (this one is kinda difficult the first time)
%%   REVTeX Guide:             http://www.physics.csbsju.edu/370/papers/Journal_Style_Manuals/auguide4-1.pdf
%%
%%   (this document is of class "revtex4-1", the REVTeX Guide explains how the class works)


%% CREATING THE .pdf FILE USING LINUX IN THE TERMINAL
%% 
%% [terminal]$ pdflatex template.tex
%%
%% Run the command twice, always.
%% If you want to use \footnote, you need to run these commands (IN THIS SPECIFIC ORDER)
%% 
%% [terminal]$ pdflatex template.tex
%% [terminal]$ bibtex template
%% [terminal]$ pdflatex template.tex
%% [terminal]$ pdflatex template.tex
%%
%% Don't ask me why, I don't know.

\begin{document}
\title{Kvantetilstand i et dobbelt-spalte system}   % self-explanatory
\author{Carl Petter Duedahl}               % self-explanatory
\author{Henrik Modahl Breitenstein}        % self-explanatory
\date{\today}                             % self-explanatory
\noaffiliation                            % ignore this
\begin{abstract}                          % marks the beginning of the abstract
Vi har sett på hvordan en kvantetilstand vil oppføre seg i en boks med én, to og tre spalter ved å gjøre numeriske beregninger. Avviket til den totale sannsynligheten fikk vi til å være på størrelsesordenen $10^{-14}$. Ved to spalter ser vi hvordan kvantetilstanden tidsutvikler seg, ved at noe reflekteres og noe går igjennom spaltene. For én, to og tre spalter ser vi på sannsynlighetsfordelingen ved $x=0.8$ og finner at sannsynlighetstetteheten ligner mye på fordelingen til interferenslinjene til bølger igjennom spalter.              % the body of the abstract
\end{abstract}                            % marks the end of the abstract
\maketitle                                % creates the title, author, date & abstract


\section{Introdukson}
Et viktig eksperiment i kvantefysikkens historie har vært dobbelt-spalte eksperimentet, først introdusert av Thomas Young på 1800-tallet (Referanse 1). Young sitt eksperiment gikk ut på hvordan lys oppfører seg som en bølge. Senere har det blitt vist hvordan partikler, som for eksempel elektroner også kan oppføre seg som bølger ved hjelp av det samme eksperimentet. Forståelsen på hvordan en partikkel interferer med seg selv er gitt av Schrödingerlikningen fra kvantefysikken:

\begin{equation}
	i \hbar \frac{d}{dt} |\Psi\rangle = \hat{H} |\Psi\rangle,
\end{equation}

For å kunne se hvordan en kvantetilstand vil oppføre seg i et slikt eksperiment så skal vi bruke Schrödingers likning til å simulere et ett-spalte, to-spalte og tre-spalte system innad i en boks. Ved å gjøre dette kan vi finne sannsynlighetsfordelingen til en tenkt partikkel og sammenlikne med tidligere teoretiske og eksperimentelle resultater. I metodedelen går vi først igjennom Schrödingerlikningen og ser hvordan vi kan forenkle den. Så viser setter vi opp en initialtilstand en vegg. Deretter ser vi på Crank-Nicolson likningen og viser hvordan vi kan bruke denme til å tidsutvikle systemet. Før vi til slutt i metodedelen bruker alt dette. Vi representer resultatene så i resultatskapittelet og diskuterer de separat i diskusjonsdelen. Til slutt summerer vi opp og kommer med en konklusjon.
\section{Teori og metode}   % (optional)

\subsection{Om Schrödinger-likningen}
Vi har altså at Schrödinger-likningen kan skrives som
$$
i\hbar\frac{d}{dt}\ket{\Psi}=\hat{H}\ket{\Psi}
$$
Vi skal imidlertid se på en partikkel i et todimensjonalt system med en vegg med et høyt tidsuavhenig potensial $V(x,y)$. Da blir Schrödinger-likningen heller slik
$$
i\hbar \frac{d}{dt}\Psi(x,y,t)=-\frac{\hbar^2}{2m}(\frac{\delta^2}{\delta x^2}+\frac{\delta^2}{\delta y^2})+V(x,y)\Psi(x,y,t)
$$
I denne oppgaven vil vi være mest opptatt av hvordan sannsynligheten for å finne partikkelen på spesifike posisjoner i systemet utvikler seg over tid. Av Borns regel har vi at sannsyligheten for å finne partikkelen i en tilstand eller i vårt tilfelle posisjon på et gitt tidspunkt er
$$
p(x,y;t)=|\Psi(x,y,t)|^2=\Psi^*(x,y,t)\Psi(x,y,t)
$$
I denne simuleringen skal vi forenkle denne modellen slik at vi nå har
$$
i\frac{d u}{dt}=-\frac{\delta u}{\delta x}-\frac{\delta u}{\delta y}+v(x,y)u
$$
Her er $u$ en normalisert og dimensjonsløs kvantetilstand. Siden den er normalisert og vi går over et gitter vil da
$$
\sum p_{i,j}=\sum u_{i,j}^*u_{i,j}=1
$$
I tillegg er $u$ i et dimensjonsløst plan så vi setter $x$ og $y$ til å gå fra $0$ til $1$. Selve initialtilstanden vil bli videre diskutert i \autoref{ssec:Init}. $v(x,y)$ er også innenfor det samme enhetsløse to-dimensjonale planet og vil bli videre diskutert i \autoref{ssec:Spalte}.
\subsection{Initialtilstand}
\label{ssec:Init}
Vi trenger en initialtilstand, altså tilstanden $u(x,y,t=0)=u^0_{i,j}$. Vi skal bruke en Gaussisk initialtilstand på formen
$$
u(x,y,t=0)=\frac{1}{C}e^{-\frac{(x-x_c)^2}{2\sigma_x^2}-\frac{(y-y_c)^2}{2\sigma_y^2}+ip_x(x-x_c)+ip_y(y-y_c)}
$$
Siden dette er Gaussisk så vil $x_c$ og $y_c$ være toppunktet til $p_{i,j}$ og der det vil være mest sannsynlig at partikkelen er. $p_x$ og $p_y$ er bevegelsesmengden til partikkelen. $\sigma_x$ og $\sigma_y$ er bredden til funksjonen. $C$ er normaliseringskonstanten. Siden vi skal gjøre dette over et gitter får vi heller
$$
u_{i,j}^0=\frac{1}{C}e^{-\frac{(x_i-x_c)^2}{2\sigma_x^2}-\frac{(y_j-y_c)^2}{2\sigma_y^2}+ip_x(x_i-x_c)+ip_y(y_j-y_c)}
$$
Vi må også normalisere dette, altså at 
$$\sum_{i,j} {u^n_{i,j}}^*u^n_{i,j}=\sum_{i,j}p^n_{i,j}=1$$
Så det vi da må gjøre er å la
$$
C=\sum_{i,j}|e^{-\frac{(x_i-x_c)^2}{2\sigma_x^2}-\frac{(y_j-y_c)^2}{2\sigma_y^2}+ip_x(x_i-x_c)+ip_y(y_j-y_c)}|^2
$$
Å normalisere slikt gjør vi kun i initialtilstanden, men dersom systemet er nøyaktig nok, vil $\sum_{i,j}p$ holde seg ganske nærme 1 også når vi tidsutvikler tilstanden. Imens vi tidsutvikler koden vår vil vi summere opp alle $p^n_{i,j}$ for å se om den totale sannsynligheten holder seg jevnt rundt $1$ og dermed også kontrollere at vi gjør riktig.
\newline Hele matrisen $U$ bestående av $u_{i,j}$ vil være på størrelsen $M\cross M$ og dimensjonene vil være normalisert så laveste verdiene av $x $ og $y$ vil være $0$ og høyeste $1$. Vi vil bruke Dirichlet grensebetingelser så vi setter $u(x=0,y, t)=u(x=1,y, t)=u(x, y=0, t)=u(x,y=1, t)=0$ uansett tidssteg. Det gjør at vi egentlig ikke trenger å finne tidsutviklingen i grensene så for det meste av metoden begrenser vi $U$ til å være en $(M-2)\cross (M-2)$-matrise, med $u_{0,0}^n=u(x=0+h,y=0+h, t)$ og $u_{M-3,M-3}=u(x=1-h,y=1-h)$. Hvor $h$ da er steglengden. Vi vil da til slutt omgi matrisen med en vegg av $0$ når vi plotter $U$.
\subsection{Lage spalten} \label{ssec:Spalte}
Så trenger vi å lage en vegg og en spalteåpning. Vi skal sette spalten i midten av systemet vårt, altså har den et midtpunkt i $x=0,5$. Så skal tykkelsen på veggen være $0,02$ i x-retning. I y-retning har vi da hullene og veggene. Vi vil i starten bruke to spalter, men vil også variere mellom å bruke én spalte, tre spalter og å ikke ha noen vegg i det hele tatt. Vi tar først eksempelet med to åpninger. Da har vi først en vegg, så en åpning på $0,05$, deretter en vegg også på $0,05$, så en ny åpning på $0,05$ og til slutt en vegg som er like lang som den første veggen. Lengden på åpningene og veggene mellom åpningene vil ikke forandre seg når vi endrer antall åpninger, men veggene på sidene vil endre seg avhengig av antall åpninger vi har. For å finne lengden for endeveggene kan vi da bruke
$$
l_{endevegg}=\frac{1-(n_{slits}+n_{mellomvegger})\cdot 0.05}{2}
$$
Hvor $n_{slits}$ er antall åpninger og $n_{mellomvegger}$ er antall vegger imellom åpningene.\newline
Vi får da at
\begin{table}[H]
\centering 
	\begin{tabular}{|c|c|}
		\hline
		Antall åpninger & Endevegg \\
		\hline 
		1 & $0,475$ \\
		\hline
		2& $0,425$ \\ \hline 
		3& $0,375$ \\ \hline
		
	\end{tabular}
	\caption{Lengden til endeveggen til barrieren basert på antall spalter.}
\end{table}
Vi går da fra 0 opp til den tilhørende vegglengden og finner høyden åpningen starter i, så går vi $0,05$ opp for å finne hvor skilleveggen starter, så går vi enda $0,05$ opp for å finne hvor neste åpning starter og slik til vi når endeveggen. Vi finner da y-verdiene vi trenger og får vegger som i \autoref{fig:Wall1}, \autoref{fig:Wall2} og \autoref{fig:Wall3}.

\begin{figure}[H]
	\centering 
	\includegraphics[scale=0.1]{../Images/1Walls.pdf}
	\caption{Veggenes og åpningenes start og ender på y-aksen i tillegg til veggenes tykkelse. Veggen har her én åpning.}
	\label{fig:Wall1}
\end{figure}
\begin{figure}[H]
	\centering 
	\includegraphics[scale=0.1]{../Images/NewWalls.pdf}
	\caption{Veggenes og åpningenes start og ender på y-aksen i tillegg til veggenes tykkelse. Veggen har her to åpninger}
	\label{fig:Wall2}
\end{figure}

\begin{figure}[H]
	\centering 
	\includegraphics[scale=0.1]{../Images/NewWalls.pdf}
	\caption{Veggenes og åpningenes start og ender på y-aksen i tillegg til veggenes tykkelse. Veggen har her tre åpninger}
	\label{fig:Wall3}
\end{figure}
Vi kan fortsatt ikke generelt anta at slike posisjoner ligger nøyaktig på et punkt på posisjonsgitteret vårt, så vi vil avrunde til det nærmeste punktet. 
\newline Idéelt sett burde veggen hatt et uendelig stort potensial for at veggdelen skulle vært helt ugjennomtrengelig. Dessverre er uendelig et altfor stort tall for maskinen å regne med. Vi setter derfor potensialet der veggen er til å være $10^{10}$, og over resten av systemet vil potensialet være $0$.
\subsection{Numerisk tillnærming} 
V har da fra Schrödingerlikningen at
$$
i\frac{\delta u}{\delta t}=-\frac{\delta^2 u}{\delta x^2}-\frac{\delta^2 u}{\delta y^2}+v(x,y)
$$
eller 
$$
\frac{\delta u}{\delta t}=i\frac{\delta^2 u}{\delta x^2}+i\frac{\delta^2 u}{\delta y^2}-iv(x,y)
$$
Vi skal så bruke Crank-Nicolson tilnærming så vi starter med å approksimere venstre-siden
$$
\frac{d u}{dt}=\frac{u^{n+1}_{i,j}-u^{n}_{i,j}}{\Delta t}
$$
hvor $n$ er tidsteget vi er i.
Crank-Nicolson baser seg på forover og bakover tilnærminger. For forover har vi at
$$
\frac{u^{n+1}_{i,j}-u^{n}_{i,j}}{\Delta t}=F^{n}_{i,j}
$$ 
mens bakover har vi
$$
\frac{u^{n+1}_{i,j}-u^{n}_{i,j}}{\Delta t}=F^{n+1}_{i,j}
$$
Så kombinerer vi både forover og bakover tilnærming og får
$$
\frac{u^{n+1}_{i,j}-u^n_{i,j}}{\Delta t}=\theta F^{n+1}_{i,j}-(1-\theta)F^{n}_{i,j}
$$
slik at for $\theta=1$ har vi bakovertilnærmingen og for $\theta=0$ har vi forovertilnærmingen.
For Crank-Nicolson setter vi $\theta =\frac{1}{2}$ slik at vi får
$$
\frac{u^{n+1}_{i,j}-u^n_{i,j}}{\Delta t}=\frac{1}{2}(F^{n+1}_{i,j}-F^{n}_{i,j})
$$

\subsection{Bruke Crank-Nicolson}
Vi hadde fra Crank-Nicolsons
$$
\frac{u^{n+1}_{i,j}-u^n_{i,j}}{\Delta t}=\frac{1}{2}(F^{n+1}_{i,j}-F^{n}_{i,j})
$$
fra forrige del og bruke den på vårt system. \newline
I vårt tilfelle er 
$$
F_{i,j}=i\frac{\delta^2 u}{\delta x^2}+i\frac{\delta^2 u}{\delta y^2}-iv(x,y)u
$$
så 
$$
F^{n}_{i,j}=i\frac{\delta^2 u^n}{\delta x^2}+i\frac{\delta^2 u^n}{\delta y^2}-iv(x,y)u^n
$$
Vi bruker deretter at
$$
\frac{\delta^2 u^n}{\delta x^2}\approx\frac{u^{n}_{i+1,j}-2u^{n}_{i,j}+u^n_{i-1,j}}{\Delta x^2}
$$
Siden $i$ er den eneste som varierer i med hensyn på x, er det denne vi vil bruke for $\frac{\delta^2u^n}{\delta x^2}$. Tilsvarende får vi at
$$
\frac{\delta^2 u^n}{\delta y^2}\approx \frac{u^n_{i,j+1}-2u^n_{i,j}+u^{n}_{i,j-1}}{\Delta y^2}
$$
Så vi får da at
$$
F^n=i\begin{pmatrix}
\frac{u^{n}_{i+1,j}-2u^{n}_{i,j}+u^n_{i-1,j}}{\Delta x^2} \\ +\frac{u^n_{i,j+1}-2u^n_{i,j}+u^{n}_{i,j-1}}{\Delta y^2}-v_{i,j}u_{i,j}
\end{pmatrix}
$$
Vi går igjen tilbake til
$$
\frac{u^{n+1}_{i,j}-u^n_{i,j}}{\Delta t}=\frac{1}{2}(F^{n+1}_{i,j}-F^{n}_{i,j})
$$
og flytter over slik at vi får
$$
u^{n+1}_{i,j}-\frac{\Delta t}{2}F^{n+1}_{i,j}=u^{n}_{i,j}+\frac{\Delta t}{2}F^{n}_{i,j}
$$
Vi utvider $F$ og får
$$
\begin{matrix}
	u_{i,j}^{n+1} \\ -\frac{i\Delta t}{2\Delta x^2}(u^{n+1}_{i+1,j}-2u^{n+1}_{i,j}+u^{n+1}_{i-1,j}) \\ -\frac{i\Delta t}{2\Delta y^2}(u^{n+1}_{i,j+1}-2u^{n+1}_{i,j}+u^{n+1}_{i,j-1}) \\ \frac{i\Delta t}{2}v_{i,j}u^{n+1}_{i,j}
\end{matrix}=\begin{matrix}
u^n_{i,j}\\ +\frac{i\Delta t}{2\Delta x^2}(u^{n}_{i+1,j}-2u^{n}_{i,j}+u^{n}_{i-1,j}) \\ +\frac{i\Delta t}{2\Delta y^2}(u^n_{i,j+1}-2u^n_{i,j}+u^{n}_{i,j-1}) \\ -\frac{i\Delta t}{2}v_{i,j}u^n_{i,j} 
\end{matrix}
$$
Vi skal gå over samme steglengde på x og y aksen så vi setter $\Delta x=\Delta y=h$. Så definerer vi $r\equiv \frac{i\Delta t}{2h^2}$ slik at vi har
$$
\begin{matrix}
	u_{i,j}^{n+1} \\ 
	-r(u^{n+1}_{i+1,j}-2u^{n+1}_{i,j}+u^{n+1}_{i-1,j}) \\ -r(u^{n+1}_{i,j+1}-2u^{n+1}_{i,j}+u^{n+1}_{i,j-1}) \\ \frac{i\Delta t}{2}v_{i,j}u^{n+1}_{i,j}
\end{matrix}=\begin{matrix}
	u^n_{i,j}\\ +r(u^{n}_{i+1,j}-2u^{n}_{i,j}+u^{n}_{i-1,j}) \\ +r(u^n_{i,j+1}-2u^n_{i,j}+u^{n}_{i,j-1}) \\ -\frac{i\Delta t}{2}v_{i,j}u^n_{i,j} 
\end{matrix}
$$
\subsection{Matriseform}
For å gjøre det litt raskere skal vi konvertere om til matriseform. For da å tidsutvikle $U$, må vi få den over på en vektorform. Vi vil derfor lage en vektor $\vec{u}$ som organiserer matrisen $U$ slik
$$
\vec{u}=(u_{0,0}, u_{1,0}, u_{2,0} (...) u_{M-2, 0} u_{0,1},(...) u_{0, M-2}, (...) u_{M-2, M-2})
$$
Hvis vi nå snakke om et element i vektoren $\vec{u}$ vil vi bruke notasjonen $u_k$, mens for matriseelementen vil vi bruke notasjonen $u_{i,j}$. Vi har da at $u_k=u_{i,j}$. Da er $k=i+j\cdot (M-2)$. Det betyr at $\vec{u}$ er $(M-2)^2$ stor.
\newline 
Vi skal så lage matrisene $A$ og $B$ slik at
$$
B\vec{u}^n=\vec{c}
$$
og
$$
A\vec{c}=\vec{u}^{n+1}
$$
La oss ta et eksempel for et lavt tall $(M-2)=3$. Matrisen for $A$ og $B$ vil da være
\begin{equation}\label{eq:A}
	A=\begin{pmatrix}
		a_0 & -r & 0 & -r & 0 &0&0&0&0 \\
		-r & a_1 & -r & 0 & -r& 0&0&0&0 \\
		0 &-r &a_2&0&0&-r&0&0&0 \\
		-r &0&0&a_3&-r&0&-r&0&0 \\
		0&-r&0&-r&a_4&-r&-r&0&0 \\
		0&0&-r&0&-r&a_5&0&0&-r \\
		0&0&0&-r&0&0&a_6&-r&0 \\
		0&0&0&0&-r&0&-r&a_7&-r \\
		0&0&0&0&0&-r&0&-r&a_8
		\end{pmatrix}
\end{equation}
og
\begin{equation}\label{eq:B}
B=\begin{pmatrix}
	b_0 & r & 0 & r & 0 &0&0&0&0 \\
	r & b_1 & r & 0 & r& 0&0&0&0 \\
	0 &r &b_2&0&0&r&0&0&0 \\
	r &0&0&b_3&r&0&r&0&0 \\
	0&r&0&r&b_4&r&r&r&0 \\
	0&0&r&0&r&b_5&0&0&r \\
	0&0&0&r&0&0&b_6&r&0 \\
	0&0&0&0&r&0&r&b_7&r \\
	0&0&0&0&0&r&0&r&b_8
\end{pmatrix}
\end{equation}
Her er diagonalene satt sammen av vektorene $\vec{a}$ og $\vec{b}$ der elementene er gitt som
$$
a_k=1+4r+\frac{i\Delta t}{2}v_{i,j}
$$
og
$$
b_k=1-4r-\frac{i\Delta t}{2}v_{i,j}
$$
Av disse matrisene kan vi se to ting. Foruten om diagonalen er $A=-B$. I tillegger matrisene satt sammen av to typer $M-2 \cross M-2$ matriser, ser man igjen bort fra diagonalene. Diagonalen til $B$ består av matrisen $P$ som har side-diagonalene $r$. For $M-2=3$ får vi da at
$$
P=\begin{pmatrix}
	0 &r&0\\
	r &0 &r \\
	0 & r& 0
\end{pmatrix}
$$
Denne matrisen vil gå diagonalt ned $B$. Som sidediagonaler til denne matrisen, altså under og til høyre for $P$ vil vi ha matrisen $R$ som har diagonalen bestående av $r$. Så for $M-2=3$ har vi da
$$
R=\begin{pmatrix}
	r&0&0 \\
	0&r&0 \\
	0&0&r
\end{pmatrix}
$$
Så da har vi uten å ta hensyn til diagonalen at
$$
-A=B=\begin{pmatrix}
	P&R&0 \\
	R&P&0 \\
	0&R&P
\end{pmatrix}
$$
Legger vi så til vektorene $\vec{a}$ og $\vec{b}$ langs diagonalene har vi matrisene $A$ og $B$ som de er gitt i \autoref{eq:A} og \autoref{eq:B}.
\subsection{Startverdier og tilhørende simuleringer}
Vi har nå laget en metode som gir oss en del resultater og er avhengig av en del verdier som den må bruke. En forklaring for alle disse finnes i \ref{table:Initverdier}
\begin{table}[H]
\centering
	\begin{tabular}{|c|p{40mm}|}
		\hline
		$h$ & Steglengden over x og y-aksen \\\hline 
		$\Delta t$ & Tidsstegene \\\hline 
		$T$ & Den totale tiden vi kjører simuleringen over. \\\hline
		$x_c$ og $y_c$ & Hvor initialtistandens sansynlighetsfordeling vil være sentrert på deres tilhørende akse. \\\hline 
		$\sigma_x$ og $\sigma_y$ & Bredden til den gaussiske funskjonen intialtilstanden består av. \\\hline 
		$p_x$ og $p_y$ & Bevegelsesmengden i x og y retning for initialtilstanden. \\\hline 
		$v_0$ & Potensialet i veggen. \\\hline
		$n_{slits}$ & Antall åpninger i veggen \\\hline
	\end{tabular}
\caption{En forklaring for alle verdiene som trengs for å kjøre simulasjonen}
\label{table:Initverdier}
\end{table}
I alle tilfeller vi tester vil $h=0,005$ $\Delta t=2,5\cdot 10^{-5}$, $x_c=0,25$, $y_c=0,5$, $\sigma_x=0,05$ og $p_y=0$. De andre vil vi variere.
\subsection{Sannsynlighetsunøyaktighet}
Som sagt tidligere burde den totale sansynligheten for å finne partikkelen i systemet holde seg ganske konstant rundt 1. Altså
$$
P^n=\sum_{i,j}p^n_{i,j}=1
$$
Hvor $P$ her er den totale sannsynligheten for å finne partikkelen i systemet.
\newline
Siden vi gjør  dette numerisk så kan vi få litt avvik fra 1, men jo større dette avviket er, jo verre er modellen vår, så avviket er en god indikasjon på hvor god modellen vår er.  Vi vil derfor som en kontrolltest, sette $v_0=0$ slik at vi ikke har en vegg. Så tester vi med $\sigma_y=0,05$, altså lik som $\sigma_x$. Vi vil da ta $P^n$ for hvert tidssteg og etterpå plotte avviket, altså $P^n-1$. Vi plotter opp til $T=0,008$. Slik har vi en kontrolltest uten noen hindring, så vi ser om sannsynligheten holder seg konstant når partikkelen går uforstyrret uten en vegg.
\newline Vi legger så til en dobbeltspaltevegg, så $s_l=2$ og $v_0=10^{10}$. Hvis vi nå fortsetter å ha $\sigma_y=0,05$ vil det meste av bølgefunksjonen gå rett på veggen med samme bredde. Vi setter derfor opp $\sigma_y$ til $0,1$ så en større del av bølgefunksjonen går gjennom spalten. Så kjører vi simulasjonen på nytt og plotter igjen avviket for denne simulsjonen også.
\subsection{Simulasjonen i det to dimensjonale tommet}\label{ssec:sim2}
Vi skal så se på $p_{i,j}$ og $u_{i,j}$ i planet. Vi vil her se på noen øyblikksbilder i $t=0$, $t=0,001$ og $t=0,002$. Vi har fortsatt en dobbeltspaltevegg med $v_0=10^{10}$, men denne gangen har vi $\sigma_y=0,2$.
Vi vil for disse tidsstegene plotte den reelle og den imaginære delen av $u_{i,j}$ i planet, i tilleg til $p_{i,j}=u*_{i,j}u_{i,j}$.
\subsection{Sannsynligheten over y-aksen for en gitt x-verdi}
Vi bruker nå de samme startverdiene som i \autoref{ssec:sim2}. La oss nå anta at ved $t=0,002$ så måler vi at partikkelens posisjon på x-aksen er $x=0,8$, men vi vet fremdeles ikke partikkelens posisjon på y-aksen. Altså bryter tilstanden sammen til at $p(x=0,5)=1$. Vi vil så finne ut hvordan sansynlighetsfordelingen på y-aksen ved $x=0,8$ vil være. Vi bruker forrige simulasjons verdier i $x=0,8$ og normaliserer disse slik at $\sum_{i}u^*_iu_i=1$. Så plotter vi over y-aksen. Vi kjører så simulasjonen igjen for $sl=1$ og $sl=3$ og finner sannsynlighetsfordelingen over y-aksen i $x=0,8$ for disse verdiene av $sl$ også. 
\section{Resultater}
Avviket til den totale sannsynligheten er vist i \autoref{Fig:ToTP} og \autoref{Fig:ToTPs1}

\begin{figure}[H]
	\centering
	\includegraphics[scale=0.4, trim={0cm 0 0 0}]{../Images/7P199.pdf}
	\caption{Avviket til den totale sannsynligheten $1 - P$ for $v_0 = 0 $.}
	\label{Fig:ToTP}
\end{figure}

\begin{figure}[H]
	\centering
	\includegraphics[scale=0.4, trim={0 0 0 0}]{../Images/7s1P199.pdf}
	\caption{Avviket til den totale sannsynligheten $1 - P$ for $v_0 = 10^10$.}
	\label{Fig:ToTPs1}
\end{figure}

For to-spalte system så fikk vi \autoref{Fig:s2u2t0} for $t = 0 \; s$, med den reelle delen vist i \autoref{Fig:s2u2t0Re} og imaginær i \autoref{Fig:s2u2t0Im}.

\begin{figure}[H]
	\centering
	\includegraphics[scale=0.45, trim={4cm 0 0 0}]{../Images/ImshowUt00sl2.pdf}
	\caption{$p=|u|^2$ i tiden $t = 0 \; s$.}
	\label{Fig:s2u2t0}
\end{figure}


\begin{figure}[H]
	\centering
	\includegraphics[scale=0.45, trim={4cm 0 0 0}]{../Images/ImshowRe00sl2.pdf}
	\caption{$Re(u) $ i tiden $t = 0 \; s$.}
	\label{Fig:s2u2t0Re}
\end{figure}


\begin{figure}[H]
	\centering
	\includegraphics[scale=0.45, trim={0cm 0 0 0}]{../Images/ImshowIm00sl2.pdf}
	\caption{$Im(u) $ i tiden $t = 0 \; s$.}
	\label{Fig:s2u2t0Im}
\end{figure}


Ved tiden $t = 0,001 \; s$ fikk vi \autoref{Fig:s2u2t01} for $p$, \autoref{Fig:s2Ret01} for den reelle delen av $u$ og \autoref{Fig:s2Imt01} for den imaginære delen.

\begin{figure}[H]
	\centering
	\includegraphics[scale=0.45, trim={1cm 0 0 0}]{../Images/ImshowUt0001sl2.pdf}
	\caption{$p=|u|^2$ i tiden $t = 0,001 \; s$.}
	\label{Fig:s2u2t01}
\end{figure}

\begin{figure}[H]
	\centering
	\includegraphics[scale=0.45, trim={4cm 0 0 0}]{../Images/ImshowRe0001sl2.pdf}
	\caption{$Re(u) $ i tiden $t = 0,001 \; s$.}
	\label{Fig:s2Ret01}
\end{figure}

\begin{figure}[H]
	\centering
	\includegraphics[scale=0.45, trim={4cm 0 0 0}]{../Images/ImshowIm0001sl2.pdf}
	\caption{$Im(u) $ i tiden $t = 0,001 \; s$.}
	\label{Fig:s2Imt01}
\end{figure}


Ved tiden $t = 0,002 \; s$ fikk vi \autoref{Fig:s2u2t02} for sannsynlighetsfordelingen og \autoref{Fig:s2Ret02} for den relle delen og \autoref{Fig:s2Imt01} for den imaginære:

\begin{figure}[H]
	\centering
	\includegraphics[scale=0.45, trim={-1cm 0 0 0}]{../Images/ImshowUt0002sl2.pdf}
	\caption{$p=|u|^2$ i tiden $t = 0,002 \; s$.}
	\label{Fig:s2u2t02}
\end{figure}

\begin{figure}[H]
	\centering
	\includegraphics[scale=0.45, trim={-1cm 0 0 0}]{../Images/ImshowRe0002sl2.pdf}
	\caption{$Re(u) $ i tiden $t = 0,002 \; s$.}
	\label{Fig:s2Ret02}
\end{figure}

\begin{figure}[H]
	\centering
	\includegraphics[scale=0.45, trim={3cm 0 0 0}]{../Images/ImshowIm0002sl2.pdf}
	\caption{$Im(u) $ i tiden $t = 0,002 \; s$.}
	\label{Fig:s2Imt02}
\end{figure}

Med én åpning så fikk vi sannsynlighetsfordelingen langs y-aksen for $x = 0.8$ i \autoref{Fig:Ps1}

\begin{figure}[H]
	\centering
	\includegraphics[scale=0.4]{../Images/ScreenProb1Slit.pdf}
	\caption{Sannsnynlighetsfordelingen ved $x = 0.8$ og $t = 0.002 \; s$ for en spalte}
	\label{Fig:Ps1}
\end{figure}

For to åpninger fikk vi \autoref{Fig:Ps2} og for tre fikk vi \autoref{Fig:Ps3}


\begin{figure}[H]
	\centering
	\includegraphics[scale=0.4]{../Images/ScreenProb2Slit.pdf}
	\caption{Sannsnynlighetsfordelingen ved $x = 0.8$ og $t = 0.002 \; s$ for to spalter}
	\label{Fig:Ps2}
\end{figure}


\begin{figure}[H]
	\centering
	\includegraphics[scale=0.4]{../Images/ScreenProb3Slit.pdf}
	\caption{Sannsnynlighetsfordelingen ved $x = 0.8$ og $t = 0.002 \; s$ for tre spalter}
	\label{Fig:Ps3}
\end{figure}

\section{Diskusjon}
\subsection{Sannsynlighetsavvik}
Av \autoref{Fig:ToTP} og \autoref{Fig:ToTPs1} ser vi at avviket for den totale sannsynligheten er på en skala av $10^{-14}$ både med og uten vegg, som forsterker troen på at vi har gjort det riktig, siden den totale sannsynligheten viker så lite fra 1. $10^{-14}$ begynner også å nærme seg maskinpresisjonen for doubles som vi brukte, altså $10^{-15}$, så det blir vanskelig å få mere nøyaktighet. Vi ser også av \ref{Fig:ToTP} at avviket ser ut til å ha en syklus, men kommer seg fortsatt ikke opp mot 0 igjen ved $t=T=0,008$. Dette kan tyde på at avviket blir større, men det kan også hende at avviket blir mindre igjen etter $t=0,008$, men siden vi bare har målt til $t=0,008$ er dette vanskelig å si. I vårt tidsrom er fortsatt avviket lite nok til at man kan si dette er akseptabelt. I \autoref{Fig:ToTPs1} ser det imedlertid ut til at avviket stabiliserer seg ved $t=0,002$. Vi vet igjen ikke hvordan den vil oppføre seg etter $t=0,008$, men for tidsperioden av vår simulering er dette akseptabelt. Så ut ifra sannsynlighetsavviket kan det tyde på at vi har ganske riktige resultater. 
\subsection{Målingene av u i planet}
Av \autoref{Fig:s2u2t0} ser vi en ganske normal todimensjonal sannsynlighetsfordeling med et toppunkt i $x=x_c, y=y_c$. Ser vi imidlertid på den relle delen og den imaganiære delen hver for seg, som i \autoref{Fig:s2u2t0Re} og \autoref{Fig:s2u2t0Im}, ser vi at $|u|^2=p$ egentlig er satt sammen av to bølgefunksjoner som ser ut til å svinge langs $x$-aksen og har en bredde langs $y$-aksen. Ser vi kun på $p$ ser vi at i $t=0,001$ så går noe av $p$ gjennom åpningene, mens noe blir reflektert i veggen og går motsatt vei, som bølger. Til slutt ved $t=0,002$ ser vi i \autoref{Fig:s2u2t02} at $p$ har blitt splittet opp til mindre toppunkter rundt om i planet. De fleste og mest sannsynlige punktene er igjen på venstre side, mens noe har fortsatt gått over til høyre side.
\newline
Ser vi derimot på den reelle og den imaginære delen i \autoref{Fig:s2Ret01} og \autoref{Fig:s2Imt01} ser vi her at det ikke bare er en stråle som $p$ på samme tidspunkt, men at det faktisk her er som bølger som går gjennom en åpning. Når vi også ser på den relle delen og den imaginære delen til $u$ for $t=0,002$ i \autoref{Fig:s2Ret02} og \autoref{Fig:s2Imt02}, så ser vi at det er bølger som har spredd seg, med noen interferenslinjer som er de som synes i \autoref{Fig:s2u2t02}.
\subsection{Sannsynligheten i $x=0,8$}
Vi ser at \autoref{Fig:Ps1} har ett toppunkt, \autoref{Fig:Ps2} har tre og \autoref{Fig:Ps3} har 5. Disse ligner veldig mye på interferenslinjene til bølger som går gjennom spalter som kan forsterke påstanden om at simulasjonen er korrekt, siden vi så at $u$ var satt sammen av bølger fra den relle og imaginære delen. 
\subsection{Andre feilkilder}
Som sagt var ikke veggens potensial uendelig stor og derfor var ikke veggen helt ugjennomtrengelig. Dette kan gjøre at vi miste noe nøyaktighet ved at noe av bølgefunksjonen kommer seg inn i veggen og det blir da en liten sannsynlighet for at partikkelen er der også. Likevel var veggens potensial ganske stor, så unøyaktigheten dette bidro med var nok minimal.
\newline I tillegg kunne vi kanskje ha minket $h$ og $\Delta t$ slik at vi kunne hatt mer presisjon under utviklingen av tilstanden. Det ser fortsatt ut til at vi har fått vist egenskapene vi ville vise av bølgefunksjonen, men hvis vi skulle gjøre dette for faktiske verdier, kan det hende denne simulasjonen ville ha blitt litt unøyaktig avhengig av hvor gode resultater man vil ha.
\section{Konklusjon}
Vi har simulert oppførelsen til en kvantetilstand i en boks. Ved å gi en initialhastighet så har vi sett hvordan en, to og tre spalter påvirker kvantetilstanden. Avviket til den totale sannsynligheten i beregningene er på $10^{-14}$. I en boks med dobbeltspalte har vi sett hvordan den imaginære og relle delen av kvantetilstanden er to bølgefunksjoner som kavntetilstanden besttår av. Sannsynlighetsfordelingen i dobbeltspalte-oppsettet har vi sett splitter seg opp ved at noe går igjennom spaltene imens en annen del blir reflektert av barrieren. For én, to og tre åpninger i veggen har vi sett nærmere på sannsynlighetsfordelingen ved $x = 0,8$, hvor vi får en fordeling som ligner mye på interferensmønsteret for vanlige bølger igjennom spalter.
% acknowledgements (optional)



%% When it comes to the bibliography I personally generate it using BibLaTeX. (see the link above if you're interested)
%% You're obviously allowed to create the references section however you like.
%% I'll keep it simple here.
\section{Referanser}  % the asterisk () after \section makes the section numbering go away
\begin{itemize}
	\item[-] APS Physics (2021, 10. Desember) \emph{Thomas Young and the Nature of Light}. \url{https://www.aps.org/publications/apsnews/200805/physicshistory.cfm }
	\item[-] Github Carl Peter Duedahl. FYS3150. \url{https://github.com/carlpd/FYS3150 }
\end{itemize}


\end{document}
